\documentclass{../tuda-exercise}

% Title information
\version{03. Juni 2021}
\sheetnumber{0}

% Fix url underfull hbox
\Urlmuskip=0mu plus 10mu

\setcounter{task}{-1}

\begin{document}

  \maketitle

  \begin{note}[title=Information:]
    Dieses Übungsblatt legt den Grundstein für alle weiteren Java Übungsblätter. Von Ihnen wird
    erwartet, dass Sie sich intensiv mit dem Blatt beschäftigen, da wir alle hier beschriebenen
    Formalitäten auf allen weiteren Übungsblättern als gegeben voraussetzen.
  \end{note}

  \begin{task}{Ist IntelliJ starbereit?}
    Importieren Sie die Vorlage \inlinejava{V00} aus Moodle und führen Sie die Klasse
    \inlinejava{IntelliJReady.java} aus. Wenn Sie alles korrekt installiert haben, wird Ihnen in
    der Konsole unten Ihre installierte Java Version ausgegeben. Achten Sie darauf, dass hier
    die Version 17 in Verbindung mit Temurin (AdoptOpenJDK) ausgegeben wird.

    Die Ausgabe der Konsole kann folgendermaßen aussehen:

    \begin{note}[color=tuda-gray]
      java.home = C:\(\backslash\)Program Files\(\backslash\)Eclipse Foundation\(\backslash\)jdk-17.0.3.7-hotspot

      java.version = 17.0.3

      java.runtime.version = 17.0.3+7
    \end{note}
  \end{task}

  \clearpage

  \begin{task}[credit=\stars{1}{3}]{Erste Schritte mit FopBot}
    Öffnen Sie nun die Klasse \inlinejava{FirstStepsBot.java}. Dort finden Sie eine Stelle,
    welche mit \inlinejava{TODO} gekennzeichnet ist. Fügen Sie hier Ihren Code ein, der folgendes
    umsetzt:

    \begin{enumerate}
      \item Erstellen Sie einen Roboter namens \inlinejava{alice},  der auf der Position (4,4)
      steht und nach rechts blickt. Er hat zu Beginn drei Coins in seiner Tasche.
      \item Lassen Sie \inlinejava{alice} nun zwei Schritte nach vorne laufen.
      \item Drehen Sie \inlinejava{alice} nun so, dass er nach oben blickt.
      \item Lassen Sie \inlinejava{alice} einen Schritt nach vorne laufen.
      \item Legen Sie einen Coin von \inlinejava{alice} ab.
      \item Lassen Sie \inlinejava{alice} zwei Schritte nach vorne laufen.
      \item Legen Sie zwei Coins mit \inlinejava{alice} ab.
      \item Drehen Sie \inlinejava{alice} nun so, dass er nach links blickt.
      \item Lassen Sie \inlinejava{alice} zwei Schritte nach vorne laufen.
      \item Lassen Sie \inlinejava{alice} den Coin aufheben.
      \item Lassen Sie \inlinejava{alice} einen Schritt nach vorne laufen.
    \end{enumerate}

    \br

    \begin{note}[title=Hinweis:, color=tuda-orange]
      Sie können durch den folgenden Verweisen mehr über die Funktionsweise von FopBot erfahren:

      \begin{itemize}
        \item Sourcecode: \url{https://github.com/FOP-2022/FOPBot}
        \item Dokumentation: \url{https://fop-2022.github.io/FOPBot/fopbot/package-summary.html}
      \end{itemize}

      Um mögliche Verständnisprobleme zu klären kann sich auch ein Blick in die Dokumentation und
      den Sourcecode lohnen.
    \end{note}

    \clearpagesolution

    \begin{solution}
      \lstinputlisting[style=Java]{codes/V4_Solution.java}

      \begin{note}[title=Information:]
        Die \inlinejava{Direction} eines FopBot wird mithilfe von einer Enumeration umgesetzt.
        Zugriff auf die jeweiligen Richtungen erhält man durch \inlinejava{Direction.X}, wobei
        für \inlinejava{X} jeweils \inlinejava{UP}, \inlinejava{DOWN}, \inlinejava{LEFT} und
        \inlinejava{RIGHT} eingesetzt werden kann. Damit man nicht jedes Mal
        \inlinejava{Direction} schreiben muss, kann man folgendes importieren:

        \begin{center}
          \inlinejava{import static fopbot.Direction.*;}
        \end{center}

        Mehr Information dazu kann mithilfe des folgenden Verweises gefunden werden:

        \begin{itemize}
          \item \url{https://docs.oracle.com/javase/tutorial/java/javaOO/enum.html}
        \end{itemize}
      \end{note}
    \end{solution}
  \end{task}

  \clearpage

  \begin{task}[credit=\stars{2}{3}]{Quadrat}
    Öffnen Sie nun die \inlinejava{KlasseSquare.java}. Dort finden Sie eine Stelle, welche mit
    \inlinejava{TODO} gekennzeichnet ist. Fügen Sie hier Ihren Code ein, der folgendes umsetzt:

    \br

    Zu Beginn platzieren Sie zwei Roboter in der Welt, von denen beide \inlinejava{20} Coins
    besitzen. Der erste Roboter befindet sich in Position \inlinejava{(0,0)} und blickt nach
    rechts, der andere befindet sich in Position \inlinejava{(9, 9)} und blickt nach links. Ihre
    Aufgabe ist es nun, ein (nicht ausgefülltes) Quadrat mithilfe der beiden Roboter, durch
    abgelegen von Coins, zu zeichnen. Dabei soll sich am Ende des Programms jeder Roboter im
    Startpunkt des jeweils anderen befinden. In Abbildung \ref{fig:V5} finden Sie einen
    Vorher-Nachher-Vergleich dieser Situation.

    \begin{figure}[h]
      \centering
      \scalebox{0.6}{
        \begin{FOPBotWorld}{10}{10}
          \path (0,0) pic[rotate=180] {Trianglebot};
          \path (9,9) pic {Trianglebot};
        \end{FOPBotWorld}
        \hspace*{-2.25cm}
        \begin{FOPBotWorld}{10}{10}
          \foreach \x/\y in {
              {0/0},
              {0/1},
              {0/2},
              {0/3},
              {0/4},
              {0/5},
              {0/6},
              {0/7},
              {0/8},
              {0/9},
              {9/0},
              {9/1},
              {9/2},
              {9/3},
              {9/4},
              {9/5},
              {9/6},
              {9/7},
              {9/8},
              {9/9},
              {1/0},
              {2/0},
              {3/0},
              {4/0},
              {5/0},
              {6/0},
              {7/0},
              {8/0},
              {1/9},
              {2/9},
              {3/9},
              {4/9},
              {5/9},
              {6/9},
              {7/9},
              {8/9},
          }{
            \putcoin{\x}{\y}{1}
          }
          \path (0,0) pic[rotate=270] {Trianglebot};
          \path (9,9) pic[rotate=90] {Trianglebot};
        \end{FOPBotWorld}
      }
      \caption{Vorher-Nachher-Vergleich}
      \label{fig:V5}
    \end{figure}

    \begin{note}[title=Verbindliche Anforderung:, color=tuda-orange]
      Das Laufen und Ablegen von Coins darf nur innerhalb einer Schleife umgesetzt werden, in der
      in jedem Durchlauf jeder der Roboter genau einen Coin ablegt! Lediglich das Drehen der
      Roboter darf außerhalb einer Schleife geschehen.
    \end{note}

    \clearpagesolution

    \begin{solution}
      \lstinputlisting[style=Java]{codes/V5_Solution.java}
    \end{solution}
  \end{task}
\end{document}
