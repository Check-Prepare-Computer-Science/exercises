\documentclass{../tuda-exercise}

% Title information
\version{20. November 2021}
\sheetnumber{6}

\begin{document}

  \maketitle

  \begin{task}[credit=\stars{0}{3}]{Theoriefragen}
    Erklären Sie kurz in eigenen Worten die folgenden Konzepte:

    \begin{enumerate}
      \item Was sind Funktionen höherer Ordnung? Wo liegen ihre Vorteile?
      \item Was ist ein Lambda-Ausdruck?
      \item Wieso sollte man Abstraktion beim Programmieren verwenden?
    \end{enumerate}

    \begin{solution}
      \begin{enumerate}
        \item Funktionen höherer Ordnung sind Funktionen, die als Parameter eine Funktion
        erhalten und oder eine Funktion als Rückgabewert haben. Ihre Vorteile liegen darin, dass
        sie ihre eigene Funktion anhand des Parameters anpassen (größere Abstraktion) oder
        Funktionen anhand bestimmter Parameter erstellen können. Durch die Abstraktion ergibt
        sich eine breitere Verwendungsmöglichkeit, d.h. eine Anpassbarkeit an eine Vielzahl
        gleichartiger Aufgaben.
        \item Lambda-Ausdrücke sind Literale von Funktionstypen und haben daher, wie Literale
        anderer Typen erst einmal keinen Namen (anonyme Funktion), solange man sie nicht einer
        Variablen/Konstanten zuweist.

        \br

        In Bezug auf Java kann man sich einen Lambda-Ausdruck folgendermaßen vorstellen: Ein
        Lambda-Ausdruck ist eine anonyme Funktion, welche nur durch Referenzen und Verweise
        angesprochen werden kann, aber nicht über einen Bezeichner.
        \item Durch das Unterteilen in Unterklassen und Methoden muss bei gleichartigen Aufgaben
        nicht bei jeder dieser Aufgabe alles von Grund auf neu programmiert werden, sondern es
        wird das Gemeinsame in eine einzige Entität (Klasse, Interface oder Methode in Java;
        Funktion in Racket) heraus faktorisiert. Es müssen also nur noch die spezifischen Aspekte
        bei jeder Aufgabe neu implementiert werden. Durch dieses Unterteilen können dann auch in
        Zukunft zusätzlich Aufgaben von einem ähnlichen Typ gelöst werden, die heute noch gar nicht
        bekannt sind. Der Hauptvorteil ist, dass man Wiederholungen vermeidet, was nicht nur
        aufwändig, sondern auch fehleranfällig ist. Ein weiterer Vorteil ist, dass man mit etwas
        Übung in abstraktem Denken einen solchen Code besser versteht und ihn daher auch besser
        weiter pflegen kann.
      \end{enumerate}
    \end{solution}
  \end{task}

  \clearpagesolution

  \begin{task}[credit=\stars{1}{3}]{Ausdrücke mit Funktionen höherer Ordnung}
    In der Vorlesung haben Sie die Funktionen \inlineracket{my-map}, \inlineracket{my-filter} und
    \inlineracket{my-fold} kennengelernt und diese selbst implementiert. Alle drei Funktionen
    gibt es mit identischer Funktionalität bereits vordefiniert in DrRacket und haben die Namen
    \inlineracket{map}, \inlineracket{filter} und
    \inlineracket{foldr}.

    \br

    Was liefern die folgenden Ausdrücke zurück? Arbeiten Sie hier ausschließlich mit Stift und
    Papier und verwenden Sie DrRacket erst hinterher, nur zur Überprüfung Ihrer Ergebnisse.

    \begin{enumerate}
      \item \inlineracket{(map + (list 1 2 3) (list 4 5 6))}
      \item \inlineracket{(filter positive? (list 1 -2 3 4 -5))}
      \item \inlineracket{(foldr + 0 (list 5 -9 3 2 5 6))}
      \item \inlineracket{(filter string? (list 1 2} \code{\textcolor{stringcolor}{"'3"' "'4"'})}
      \item \inlineracket{(first (map list (list}  \code{\textcolor{stringcolor}{"'x"' "'y"'
      "'z"'})))}
      \item \inlineracket{(map list (list}  \code{\textcolor{stringcolor}{"'a"' "'b"' "'c"'})}
      \inlineracket{(list 1 2 3) (list true false true))}
      \item \inlineracket{(foldr cons (list -10 -1) (list 1 10 100 1000))}
    \end{enumerate}

    \begin{solution}
      \begin{enumerate}
        \item \inlineracket{(list 5 7 9)}
        \item \inlineracket{(list 1 3 4)}
        \item \inlineracket{12}
        \item \inlineracket{(list}  \code{\textcolor{stringcolor}{"'3"' "'abc"'})}
        \item \inlineracket{(list}  \code{\textcolor{stringcolor}{"'x"''})}
        \item \inlineracket{list (list}  \code{\textcolor{stringcolor}{"'a"'}} \inlineracket{1
        true)}
        \code{\textcolor{stringcolor}{"'b"'}} \inlineracket{2 false)}
        \code{\textcolor{stringcolor}{"'c"'}}
        \inlineracket{3 true)}
        \item \inlineracket{(list 1 10 100 1000 -10 -1)}
      \end{enumerate}
    \end{solution}
  \end{task}

  \clearpagesolution

  \begin{task}[credit=\stars{1}{3}]{Funktionen höherer Ordnung verwenden}
    Definieren Sie die folgenden Funktionen. Außerhalb der Funktionen \inlineracket{map},
    \inlineracket{filter} und \inlineracket{foldr} darf keine Rekursion verwendet werden.
    \begin{itemize}
      \item Eine Funktion \inlineracket{zip}, die aus zwei gleich langen Listen eine Liste von
      geordneten Paaren macht. Beispiel:
      \\
      \inlineracket{(zip (list}  \code{\textcolor{stringcolor}{"'a"' "'b"'})} \inlineracket{(list
      1 2))}
      \(\rightarrow\)  \inlineracket{(list (list)} \code{\textcolor{stringcolor}{"'a"'}}
      \inlineracket{1) (list} \code{\textcolor{stringcolor}{"'b"'}} \inlineracket{2))}
      \item Eine Funktion \inlineracket{vec-mult}, die zwei gleich lange Listen von Zahlen erhält
      und das Skalarprodukt, also die Summe der paarweisen Produkte berechnet. Beispiel:
      \\
      \inlineracket{(vec-mult (list 1 2 3) (list 4 5 6))} \(\rightarrow\) \inlineracket{(+ (* 1
      4)(* 2 5)(* 3 6))} \(\rightarrow\) \inlineracket{32}
    \end{itemize}

    \begin{solution}
      \begin{itemize}
        \item \inlineracket{zip}
        \lstinputlisting[style=Racket]{codes/V3_01_Solution.rkt}
        \item \inlineracket{vec-mult}
        \lstinputlisting[style=Racket]{codes/V3_02_Solution.rkt}
      \end{itemize}
    \end{solution}
  \end{task}

  \clearpagesolution

  \begin{task}[credit=\stars{1}{3}]{Lambda-Ausdrücke}
    \lstinputlisting[style=Racket]{codes/V4_Task.rkt}

    Ergänzen Sie den Vertrag, sowohl für die Funktion \inlineracket{z}, als auch für den
    Lambda-Ausdruck. Was liefert \inlineracket{((z 3) 4)} zurück

    \begin{solution}
      \lstinputlisting[style=Racket]{codes/V4_Solution.rkt}

      Das Ergebnis von \inlineracket{((z 3) 4)} ist \inlineracket{12}.
    \end{solution}
  \end{task}

  \clearpagesolution

  \begin{task}[credit=\stars{2}{3}]{Foo Reloaded I}
    \label{task:V5}
    Erinnern Sie sich noch an die Funktion \inlineracket{foo} aus Aufgabe V7 vom letzten
    Übungsblatt? Zur Erinnerung: Gegeben ist ein Struct-Typ \inlineracket{abc} mit zwei Feldern
    \inlineracket{a} und \inlineracket{b}. Die Funktion \inlineracket{foo} bekommt einen
    Parameter \inlineracket{p} und liefert falls \inlineracket{p} vom Typ \inlineracket{abc} und
    zudem der Wert im Feld \inlineracket{b} von \inlineracket{p} eine Liste ist eine Liste
    zurück, deren erstes Element der Wert von Feld \inlineracket{a} in \inlineracket{p} ist, und
    der Rest der zurückgelieferten Liste ist die Liste im Feld \inlineracket{b} von
    \inlineracket{p} (also eine Liste in der Liste). Andernfalls liefert \inlineracket{foo}
    einfach \inlineracket{false} zurück.

    \br

    Definieren Sie nun eine Funktion \inlineracket{bar1}, die einen Parameter \inlineracket{lst}
    übergeben bekommt. Für jedes Element \inlineracket{x} in \inlineracket{lst}, das vom Typ
    \inlineracket{abc} ist, soll die Ergebnisliste von \inlineracket{bar1} das Ergebnis der
    Anwendung von \inlineracket{foo} auf \inlineracket{x} enthalten. Weitere Elemente darf die
    Ergebnisliste von \inlineracket{bar1} nicht enthalten.

    \br

    \begin{note}[title=Verbindliche Anforderung:, color=tuda-orange]
      Sie dürfen in dieser Aufgabe noch keine Funktionen höherer Ordnung wie \inlineracket{map} oder
      \inlineracket{filter} verwenden. Diese Funktionalitäten müssen von Ihnen selbst
      implementiert werden.
    \end{note}

    \begin{solution}
      \lstinputlisting[style=Racket]{codes/V5_Solution.rkt}
    \end{solution}
  \end{task}

  \begin{task}[credit=\stars{2}{3}]{Foo Reloaded II}
    Definieren Sie nun eine Funktion \inlineracket{bar2}. Diese besitzt die gleiche
    Funktionalität wie \inlineracket{bar1} aus Aufgabe \hyperref[task:V5]{V5}. In dieser Aufgabe
    wird die Funktionalität allerdings nicht mehr selbstgeschrieben, sondern an die
    vordefinierten Funktionen \inlineracket{map} und \inlineracket{filter} delegiert. Nutzen Sie
    Lambda-Ausdrücke, welche Sie innerhalb der Aufrufe von \inlineracket{map} und
    \inlineracket{filter} definieren.

    \begin{solution}
      \lstinputlisting[style=Racket]{codes/V6_Solution.rkt}
    \end{solution}
  \end{task}

  \clearpagesolution

  \begin{task}[credit=\stars{2}{3}]{Kartesisches Produkt}
    Definieren Sie eine Funktion \inlineracket{cartesian-prod}, die zwei Zahlenlisten erhält und das
    \href{https://de.wikipedia.org/wiki/Kartesisches_Produkt}{kartesische Produkt} der beiden bildet. Beispiel:
    \newline
    \inlineracket{(cartesian-prod (list 1 2) (list 3 4))} \(\rightarrow\)
    \inlineracket{(list (list 1 3) (list 1 4) (list 2 3) (list 2 4))}

    \begin{solution}
      \lstinputlisting[style=Racket]{codes/V7_Solution.rkt}
    \end{solution}
  \end{task}

  \clearpage

  \begin{task}[credit=\stars{3}{3}]{Bibliothek Leihgebühren}
    Eine Bibliothek verwaltet ihr Leihsystem nun in Racket. Dazu wird ein neuer Struct-Typ
    \inlineracket{br} definiert.

    \lstinputlisting[style=Racket]{codes/V8_Task.rkt}

    Das Feld \inlineracket{id} ist dabei ein String und stellt die ID-Nummer des ausgeliehenen
    Buches dar. Das Feld \inlineracket{pop} ist eine Zahl zwischen 1 bis 6 und gibt die
    Beliebtheit des Buches an (je größer die Zahl, desto beliebter das Buch). Das letzte Feld
    \inlineracket{type} ist wieder ein String, der entweder
    \code{\textcolor{stringcolor}{"'Single"'}} oder \code{\textcolor{stringcolor}{"'Subscription"'}}
    sein kann, je nachdem, ob es sich um eine einmalige Ausleihe oder einen Abonnenten handelt.

    \br

    Folgende Regeln gelten in der Bibliothek: Abonnenten zahlen für jedes ausgeliehene Buch
    pauschal 1,50€. Bei normalen Kunden berechnet sich der Preis über die Beliebtheit des Buches.
    Pro Beliebtheitsstufe kostet das Buch 1,75€. Somit kostet ein Buch mit Beliebtheitsstufe 3
    beispielsweise 5,25€.

    \br

    Ihre Aufgabe ist es nun eine Funktion \inlineracket{fee-total} zu definieren. Diese enthält
    eine Liste von \inlineracket{br}-Structs (die Ausleihliste) und gibt die Gesamteinnahmen aus
    eben dieser Ausleihliste zurück.

    \begin{solution}
      \lstinputlisting[style=Racket]{codes/V8_Solution.rkt}
    \end{solution}
  \end{task}

  \clearpagesolution

  \begin{task}[credit=\stars{3}{3}]{Wer bekommt die Zulassung?}
    Schreiben Sie eine Prozedur zur Prüfung der Zuteilung einer Studienleistung im Modul X. Dort
    sind 50 Hausaufgaben-, 35 Zwischenklausur- und 50 Projektpunkte sowie insgesamt mindestens
    180 Punkte aus den drei Bereichen zusammen erforderlich. Definieren Sie dazu eine Funktion
    \inlineracket{passed} mit folgender Signatur

    \br

    \inlineracket{(list} of \inlineracket{number)} \inlineracket{(list} of \inlineracket{(list} of
    \inlineracket{number number number))} \(\rightarrow\) \inlineracket{(list} of \inlineracket{number)}

    \br

    Diese Funktion erhält aus Datenschutzgründen separat die Liste der Matrikelnummern sowie eine
    Liste von Listen mit Punkten für Hausaufgaben, Zwischenklausur und Projekt (in dieser
    Reihenfolge). Die Precondition ist dabei, dass die Listen gleich lang sind und dass die
    Matrikelnummer an Position \inlineracket{i} der ersten Liste zu den Punkten an Position
    \inlineracket{i} der zweiten Liste gehört (vergessen Sie die Precondition nicht im Vertrag
    der Funktion). Die Ergebnisliste enthält die Matrikelnummern aller Studierenden, die die
    Bedingungen für die Studienleistung erfüllt haben. Die Reihenfolge der Studierenden soll
    dabei erhalten bleiben.

    \begin{solution}
      \lstinputlisting[style=Racket]{codes/V9_Solution.rkt}
    \end{solution}
  \end{task}

  \clearpage

  \begin{task}[credit=\stars{3}{3}]{Bildverarbeitung in Racket}
    Um diese Aufgaben in DrRacket ausführen zu können, setzen Sie bitte \inlineracket{(require
    2htdp/image)} in die oberste Zeile Ihrer Datei ein.

    \br

    Bilder bestehen aus vielen aufeinanderfolgenden Pixeln. Jedes Pixel nimmt dabei genau die
    Farbe an, die durch sein sogenanntes RGB-Tripel beschrieben werden. Dies ist durch die
    Darstellung im sogenannten \href{https://de.wikipedia.org/wiki/RGB-Farbraum}{RGB-Farbraum},
    ein sogenannter technischer Farbraum, der die Farbwahrnehmung durch das additive Mischen der
    drei Grundfarben nachbildet, begründet. Jede Farbe lässt sich dabei durch ein Tripel \textit{
      (R, G, B)} darstellen, wobei die drei Zahlen jeweils den Anteil der jeweiligen Grundfarbe
    angeben. So ist das klassische rot durch \((255,0,0)\), gelb als Mischung zweier Grundfarben
    durch \((255,255,0)\) und braun als Mischung aller Grundfarben als \((153,102,51)\) dargestellt.

    \br

    Der Einfachheit wegen benutzen wir nur Bilder im PNG-Format (d.h. Dateiendung \inlineracket{
      .png}), die keine transparenten Farben enthalten,  also keinen Alphakanal besitzen. Ein
    Bild ist in Racket immer ein Struct vom Typ
    \inlineracket{image}. Jedes Bild besteht aus seinen aufeinanderfolgenden Pixeln. In Racket
    ist ein Pixel als \inlineracket{color}-Struct definiert

    \lstinputlisting[style=Racket]{codes/V10_Task.rkt}

    Die ersten drei Felder sind das Tripel des RGB-Farbraums und liegen zwischen \(0\) und \(255\).
    Den Alpha-Wert(Siehe Transparenz, Kapitel 02 der Vorlesung) ignorieren wir in dieser Übung,
    er soll immer auf \(255\) gesetzt werden. Folgende Funktionen gibt es bereits für die
    Bildverarbeitung in Racket:

    \begin{itemize}
      \item Um aus einem \inlineracket{image} die entsprechenden \inlineracket{color}-Structs zu
      bekommen, gibt es die Funktion
      \\
      \inlineracket{(image->color-list img)}.
      \\
      Diese gibt eine Liste von \inlineracket{color}-Structs für das übergebene Bild zurück.
      \item Um aus einer Liste von \inlineracket{color}-Structs ein Bild zu generieren gibt es
      die Funktion
      \inlineracket{(color-list->bitmap clr-lst width height)}. Diese benötigt neben der Liste von
      \inlineracket{color}-Structs auch die Breite und Höhe des zu generierenden Bildes (über die
      Funktionen \inlineracket{(image-width img)} und \inlineracket{(image-height img)} abrufbar).
      \item Mit \code{(bitmap/file \textcolor{stringcolor}{"'image.png"')}} laden Sie das Bild im
      PNG-Format namens \textcolor{stringcolor}{"'image"'}, welches im gleichen Verzeichnis wie
      die \inlineracket{.rkt} Datei liegt. Mittels \code{(save-image img "' out.png"')} speichern
      Sie ein \inlineracket{image}-Struct unter dem Namen \code{\textcolor{stringcolor}{"'out"'}}
      dort.
    \end{itemize}

    Nutzen Sie für die folgenden beiden Aufgaben Funktionen höherer Ordnung

    \begin{enumerate}
      \item Definieren Sie eine Funktion \inlineracket{(count-black-white img)}. Diese bekommt
      ein Bild übergeben, welches nur aus schwarzen \((0,0,0)\) und weißen Pixeln \((255,255,255)\)
      besteht. Zurückgegeben werden soll eine zweielementige Liste, welche an erster Position
      die Anzahl an schwarzen und an zweiter Position die Anzahl an weißen Pixeln enthält.
      \item Definieren Sie eine Funktion \inlineracket{(negative-transformation img)}. Diese
      bekommt ein Bild als \inlineracket{image}-Struct übergeben und gibt die
      Negativtransformation dieses Bildes zurück. Dazu berechnen Sie die RGB-Werte für jeden
      Pixel neu über den folgenden Zusammenhang:
      \((R_{\text{neg}}, G_{\text{neg}}, B_{\text{neg}}) = (255 - R, 255 - G, 255 - B)\)
    \end{enumerate}

    \clearpagesolution

    \begin{solution}
      \lstinputlisting[style=Racket]{codes/V10_Solution.rkt}
    \end{solution}
  \end{task}
\end{document}
