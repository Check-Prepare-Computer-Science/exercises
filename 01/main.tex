\documentclass{../tuda-exercise}

% Title information
\version{20. November 2021}
\sheetnumber{1}

% Fix url underfull hbox
\Urlmuskip=0mu plus 10mu

\begin{document}

  \maketitle

  \begin{task}[credit=\stars{0}{3}]{FopBot}
    Beschreiben Sie kurz in ihren eigenen Worten worum es sich bei FopBot handelt, wie die Welt
    aufgebaut ist und
    welche Grundfunktionen jeder Roboter beherrscht.

    \begin{note}[
      title=Für alle Aufgaben auf diesem und allen weiteren Übungsblättern:,
      color=tuda-red
    ]
      In der Abschlussklausur werden Sie keine Hilfsmittel zur Verfügung haben. Üben Sie also
      schon zu Beginn auch ohne Entwicklungsumgebung und nur mit Stift und Papier zu
      programmieren. Abschließend können Sie dann ihre vollständige Lösung in die
      Entwicklungsumgebung übertragen und überprüfen.
    \end{note}

    \begin{solution}
      FopBot ist ein Package mit mehreren Klassen. Man kann damit eine beliebige Welt (mit
      Wänden) generieren und beliebig viele Objekte vom Typ Robot rein setzen, die jeweils
      beliebig viele Coins haben können. In der Klasse Robot sind mehrere Methoden, die es dem
      Robot erlauben mit der Welt zu interagieren wie bspw. move(), die den Robot in eine
      Richtung laufen lässt (eine Richtung ist mithilfe von Directions als Enumeration umgesetzt).

      \begin{note}[title=Information:]
        \enquote{Ein Package ist ein Namespace mithilfe dessen eine Reihe verwandter Klassen und
        Interfaces organsisiert werden können.}(\url{https://docs.oracle.com/
        javase/tutorial/java/concepts/package.html})

        \br

        Als Beispiel dazu das Package \inlinejava{java.awt}. Dieses Package enthält Klassen zum
        Erstellen von Benutzeroberflächen und zum Zeichnen von Grafiken und Bildern.

        \br

        Mithilfe des folgenden Verweises können Sie sich einen Überblick über Packages verschaffen:

        \begin{center}
          \url{https://docs.oracle.com/en/java/javase/11/docs/api/allpackages-index.html}
        \end{center}

        Mehr Informationen zu FopBot finden Sie unter den folgenden Verweisen:

        \begin{itemize}
          \item Sourcecode: \url{https://github.com/FOP-2022/FOPBot}
          \item Dokumentation: \url{https://fop-2022.github.io/FOPBot/fopbot/package-summary.html}
        \end{itemize}
      \end{note}
    \end{solution}
  \end{task}

  \clearpagesolution

  \begin{task}[credit=\stars{0}{3}]{Liegen geblieben}
    Betrachten Sie den folgenden Codeausschnitt/führen Sie ihn selbst einmal aus:

    \lstinputlisting[style=Java]{codes/V2_Task.java}

    \begin{solution}
      Es kommt in der Zeile 4 zu einem Fehler, weil der Roboter keinen Coin hat, den er ablegen
      kann.
    \end{solution}
  \end{task}

  \clearpagenosolution

  \begin{task}[credit=\stars{1}{3}]{Rechteck}
    Schreiben Sie ein Programm, welches zwei Roboter \inlinejava{putbot} und \inlinejava{pickbot}
    erstellt. Dabei soll \inlinejava{putbot} mit Coins ein Rechteck der Höhe \inlinejava{5} und
    der Breite \inlinejava{3} zeichnen. Es sollen nur die Seiten des Rechtecks gezeichnet werden,
    die restlichen innen liegenden Felder des Rechtecks bleiben unberührt. Nachdem das Rechteck
    gezeichnet wurde, soll \inlinejava{pickbot} alle Coins wieder einsammeln. Überlegen Sie sich,
    wie Sie das Programm mit nur einer Schleife pro Roboter gestalten können.

    \begin{figure}[h]
      \centering
      \begin{FOPBotWorld}{5}{6}
        \foreach \x/\y in {
            {0/0},
            {0/1},
            {0/2},
            {0/3},
            {0/4},
            {2/0},
            {2/1},
            {2/2},
            {2/3},
            {2/4},
            {1/0},
            {1/4},
        }{
          \putcoin{\x}{\y}{1}
        }
        \path (0,0) pic {Trianglebot};
        \path (0,0) pic[rotate=270] {Trianglebot};
      \end{FOPBotWorld}
      \caption{Fertiggestelltes Rechteck durch \inlinejava{putbot}}
    \end{figure}

    \clearpagesolution

    \begin{solution}
      \lstinputlisting[style=Java]{codes/V3_Solution.java}

      \begin{note}[title=Information:]
        Der Lösungsvorschlag verwendet in Zeile 19 und 41 \enquote{Bedingte Operatoren}. In
        diesem Fall ist es ein logisches oder, d.h. wenn Bedingung 1 oder 2 zutrifft, dann führen
        wir die folgenden Anweisungen aus.

        \begin{table}[H]
          \centering
          \begin{tabular}{ccc}
            \toprule
            Erste Bedingung & Zweite Bedingung & Ausführung der Anweisungen?
            \\\midrule
            \inlinejava{false} & \inlinejava{alse} & Nein
            \\
            \inlinejava{false} & \inlinejava{true} & Ja
            \\
            \inlinejava{true} & \inlinejava{false} & Ja
            \\
            \inlinejava{true} & \inlinejava{true} & Ja
            \\\bottomrule
          \end{tabular}
          \caption{Wahrheitstabelle zum logischen oder}
        \end{table}

        Als alternative zum logischen oder könnte man zwei unabhängige \inlinejava{if}-Abfragen
        verwenden:

        \lstinputlisting[style=Java]{codes/V3_Information.java}

        Mehr Informationen zu bedingten Operatoren kann man unter folgendem Verweis finden:

        \begin{center}
          \url{https://docs.oracle.com/javase/tutorial/java/nutsandbolts/op2.html}
        \end{center}
      \end{note}
    \end{solution}
  \end{task}

  \begin{task}[credit=\stars{1}{3}]{Bedingungen I}
    Betrachten Sie folgenden Codeausschnitt:

    \lstinputlisting[style=Java]{codes/V4_Task.java}

    Beschreiben Sie in eigenen Worten, welchem Zweck dieser Codeausschnitt dient. Erweitern Sie
    außerdem den Code so, dass \inlinejava{bot1} nur einen Coin ablegt, wenn er auch mindestens
    einen besitzt.

    \clearpagesolution

    \begin{solution}
      Es wird ein Roboter an der Position \inlinejava{(3, 1)} mit der Blickrichtung nach oben
      platziert. Dieser Roboter hat eine Münze und führt einen Schritt nach vorne (in
      Blickrichtung) aus. Falls sich unter ihm eine Münze befindet, so hebt er diese auf.
      Ansonsten legt er eine Münze ab (Erweiterung: Sofern er mindestens eine Münze besitzt).

      \lstinputlisting[style=Java]{codes/V4_Solution.java}
    \end{solution}
  \end{task}

  \begin{task}[credit=\stars{1}{3}]{Variablen}
    Legen Sie eine Variable \inlinejava{int a} an und setzen Sie ihren Wert auf \inlinejava{127}.
    Jetzt legen Sie eine weitere Variable \inlinejava{int b} an und setzen Ihren Wert auf
    \inlinejava{42}. Was gibt nun der Ausdruck \inlinejava{int c = a \% b} wieder? Beschreiben
    Sie in Ihren eigenen Worten, welche Berechnung mit dem \inlinejava{\%} Operator durchgeführt
    wird.

    \begin{solution}
      Der Ausdruck gibt 1 zurück, also den Rest der ganzzahligen Division von \inlinejava{a} und
      \inlinejava{b}. Beim \inlinejava{\%} handelt sich um den Modulo-Operator, welcher den Rest
      einer ganzzahligen Division zurückgibt.
    \end{solution}
  \end{task}

  \begin{task}[credit=\stars{1}{3}]{Bedingungen II}
    Ihr Kommilitone ist etwas tippfaul und lässt deswegen gerne einmal Klammern weg, um sich
    Arbeit zu sparen. Er hat in seinem Code eine Variable \inlinejava{int number} angelegt, in
    der er eine Zahl speichert. Ist diese Zahl kleiner als \inlinejava{0}, so möchte er das
    Vorzeichen der Zahl umdrehen und sie anschließend um \inlinejava{1} erhöhen. Ist die Zahl
    hingegen größer als \inlinejava{0}, so möchte er die Zahl verdoppeln. Dazu schreibt er
    folgenden Code:

    \lstinputlisting[style=Java]{codes/V6_01_Task.java}

    Kann der Code so ausgeführt werden? Beschreiben Sie den Fehler, den ihr Kommilitone begangen
    hat.

    \br

    Nachdem Sie ihren Kommilitonen auf den obigen Fehler hingewiesen haben, überarbeitet er
    seinen Versuch. Wie sieht es mit folgender Variante aus?

    \lstinputlisting[style=Java]{codes/V6_02_Task.java}

    Da müssen Sie wohl selbst ran. Erstellen Sie ein Codestück, um den Sachverhalt korrekt zu
    implementieren.

    \begin{solution}
      \begin{enumerate}
        \item Zwischen einer \inlinejava{if}- und \inlinejava{else}-Anweisung darf keine
        weitere Anweisung nach der Ersten zwischen dem \inlinejava{if} und
        \inlinejava{else} stehen. Der Komilitone müsste die erste und zweite Anweisung nach dem
        \inlinejava{if} in Klammern schreiben.
        \item Hier fehlen die Klammern bei der zweiten Anweisung nach dem \inlinejava{if}, da der
        \inlinejava{counter} ansonsten immer um \inlinejava{1} inkrementiert wird.

        \br

        Der verbesserte Code lautet:
      \end{enumerate}

      \lstinputlisting[style=Java]{codes/V6_Solution.java}
    \end{solution}
  \end{task}

  \clearpagesolution

  \begin{task}[credit=\stars{1}{3}]{Schleifen I}
    Schreiben Sie den folgenden Ausdruck mithilfe einer \inlinejava{for}-Schleife:

    \lstinputlisting[style=Java]{codes/V7_Task.java}

    \begin{solution}
      \lstinputlisting[style=Java]{codes/V7_Solution.java}
    \end{solution}
  \end{task}

  \begin{task}[credit=\stars{1}{3}]{Schleifen II}
    Ihr klammerfauler Kommilitone hat auch diesmal wieder zugeschlagen und versucht den Code aus
    vorheriger Aufgabe kürzer zu schreiben. Was sagen Sie dazu?

    \lstinputlisting[style=Java]{codes/V8_Task.java}

    \begin{solution}
      Der Roboter würde in diesem Falle unendlich weiterlaufen, da die Inkrementation der
      Variable \inlinejava{i} außerhalb der \inlinejava{while}-Schleife ist.

      \begin{note}[title=Information:]
        Inkrementation stammt aus dem lateinischen Wort \enquote{incrementare} (deutsch:
        vergrößern) und beschreibt im Bezug zum oberen Codeauschnitt die schrittweise
        Vergrößerung einer Variable.
      \end{note}
    \end{solution}
  \end{task}

  \begin{task}[credit=\stars{2}{3}]{Anzahl an Umdrehungen}
    Legen Sie eine Variable \inlinejava{int numberOfTurns} an und setzen Sie ihren Wert zu Beginn
    auf \inlinejava{0}. Erstellen Sie dann einen neuen Roboter und platzieren Sie ihn an der
    Stelle \inlinejava{(8,2)}. Er schaut dabei nach links und besitzt keine Coins. Lassen Sie den
    Roboter nun geradewegs auf die Stelle \inlinejava{(0,2)} zusteuern und alle Coins auf seinem
    Weg aufsammeln. Liegen mehrere Coins auf einer Stelle, so soll er alle Coins aufsammeln. Bei
    jedem Aufsammeln, erhöhen Sie den Wert von \inlinejava{numberOfTurns} um \inlinejava{1}. Hat
    er am Ende die Stelle \inlinejava{(0,2)} erreicht, soll er sich
    \inlinejava{numberOfTurns}-mal nach links drehen.

    \begin{solution}
      \lstinputlisting[style=Java]{codes/V9_Solution.java}
    \end{solution}
  \end{task}

  \begin{task}[credit=\stars{2}{3}]{Vorsicht Wand!}
    Gehen Sie in dieser Aufgabe davon aus, dass Sie einen Roboter \inlinejava{wally} an der
    Position \inlinejava{(0, 0)} erstellt haben und er nach rechts schaut. An der Position
    \inlinejava{(0, 0)} befindet sich eine vertikale Wand die den Weg nach \inlinejava{(x + 1, 0)}
    versperrt, die \inlinejava{x}-Koordinate ist allerdings unbekannt. Schreiben Sie ein
    kleines Programm, mit dem Sie den Roboter bis vor die Wand laufen lassen, direkt vor der Wand
    einen Coin ablegen, um dann wieder an die Ausgangsposition \inlinejava{(0,0)} zurückzukehren.

    \begin{note}[title=Hinweis:, color=tuda-orange]
      Es gibt die Funktion \inlinejava{isFrontClear()}, mit der getestet werden kann, ob sich in
      Blickrichtung des Roboters vor ihm direkt eine Wand befindet.
    \end{note}

    \begin{solution}
      \lstinputlisting[style=Java]{codes/V10_Solution.java}
    \end{solution}
  \end{task}

  \clearpage

  \begin{task}[credit=\stars{2}{3}]{Codeverständnis}
    Beschreiben Sie ausführlich, welches Verhalten der nachfolgende Code umsetzt. Bei Fragen zur
    Funktionalität einzelner Methoden, werfen Sie einen Blick in die entsprechenden
    Vorlesungsfolien.

    \lstinputlisting[style=Java]{codes/V11_Task.java}

    \clearpagesolution

    \begin{solution}
      \lstinputlisting[style=Java]{codes/V11_Solution.java}
    \end{solution}
  \end{task}

  \begin{task}[credit=\stars{3}{3}]{Navigator}
    Gegeben seien vier Variablen:

    \begin{minipage}{0.475\textwidth}
      \begin{flushright}
        \inlinejava{int startX}

        \br

        \inlinejava{int destinationX}
      \end{flushright}
    \end{minipage}
    \hfill
    \begin{minipage}{0.475\textwidth}
      \inlinejava{int startY}

      \br

      \inlinejava{int destinationY}
    \end{minipage}

    \br

    Ihr Roboter befindet sich zu Beginn an der Position \inlinejava{(startX, startY)} und schaut
    in eine beliebige Richtung. Schreiben Sie ein Programm, das ihn von dieser Position auf die
    Position \inlinejava{(destinationX, destinationY)} laufen lässt.

    \begin{solution}
      \lstinputlisting[style=Java]{codes/V12_Solution.java}
    \end{solution}
  \end{task}
\end{document}
