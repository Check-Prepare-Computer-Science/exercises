\documentclass{../tuda-exercise}

% Title information
\version{20. November 2021}
\sheetnumber{4}

\begin{document}

  \maketitle

  \begin{task}[credit=\stars{0}{3}]{Grundbegriffe}
    Erklären Sie kurz in eigenen Worten die Unterschiede der folgenden Konzepte zueinander:

    \begin{enumerate}
      \item Klasse vs. Objekt
      \item Objekt- vs. Klassenmethoden
      \item Abstrakte Klassen vs. Interfaces
      \item Überladen von Methoden vs. Überschreiben von Methoden
    \end{enumerate}

    \begin{solution}
      \begin{enumerate}
        \item Klasse vs. Objekt
        \begin{table}[h]
          \centering
          \begin{tabular}{p{22.5em}p{22.5em}}
            \textbf{Klasse} & \textbf{Objekt}
            \\
            Eine Klasse ist eine Beschreibung eines Objekts mit seinen Attributen und enthält die
            Definitionen aller Methoden (genauer: aller Methoden die gegenüber der Basisklasse
            hinzukommen oder zwar in der Basisklasse schon vorhanden sind, aber überschrieben
            werden). Sie dient als Vorlage, aus der dann beliebig viele Objekte erzeugt werden
            können.
            & Ein Objekt ist die Instanziierung einer Klasse mit spezifischen Werten für die
            Attribute einer Klasse.
          \end{tabular}
        \end{table}
        \item Objekt- vs. Klassenmethoden
        \begin{table}[h]
          \centering
          \begin{tabular}{p{22.5em}p{22.5em}}
            \textbf{Objektmethoden} & \textbf{Klassenmethoden}
            \\
            Objektmethoden können nur mithilfe eines Objekts aufgerufen werden.
            & Klassenmethoden können direkt über den Namen der Klasse aufgerufen werden, ohne
            dass zuvor ein Objekt der Klasse generiert wurde.
            \\
            Objektmethoden können auf alle Attribute des Objekts zugreifen und diese Lesen und
            Schreiben.
            & Klassenmethoden werden mit dem Modifier \inlinejava{static} deklariert.
            \\
            Objektmethoden sind individuell für das Objekt.
            \\
            Objektmethoden können zudem auch alle Methoden der Klasse aufrufen.
            & Klassenmethoden dürfen nur auf Klassenattribute zugreifen und nur Klassenmethoden
            aufrufen.
          \end{tabular}
        \end{table}
        \clearpage

        \item Abstrakte Klassen vs. Interfaces
        \begin{table}[h]
          \centering
          \begin{tabular}{p{22.5em}p{22.5em}}
            \textbf{Abstrakte Klassen} & \textbf{Interfaces}
            \\
            Es können keine Objekte instanziiert werden.
            & Es können keine Objekte instanziiert werden.
            \\
            Eine abstrakte Klasse kann implementierte Methoden enthalten.
            & Bei einem Interface sind die Objektmethoden entweder \inlinejava{default} oder
            nicht implementiert und die Attribute sind mit dem Modifier \inlinejava{static final}
            (Konstanten) gekennzeichnet. Zudem kann ein Interface kann auch Klassenmethoden haben.
            \\
            Die Methoden und Attribute einer abstrakten Klassen können beliebige Zugriffsmodifier
            besitzen.
            & Der Zugriffsmodifier der Methoden und Attribute eines Interfaces sind immer
            \inlinejava{public}, weshalb der Zugriffsmodifier auch weggelassen werden kann.
            \\
            Auf Klassen gibt es generell nur Einfachvererbung.
            & Ein Interface kann mehrere Interfaces erweitern und eine Klasse kann mehrere
            Interfaces implementieren.
            \\
          \end{tabular}
        \end{table}
        \item Überladen von Methoden vs. Überschreiben von Methoden
        \begin{table}[H]
          \centering
          \begin{tabular}{p{22.5em}p{22.5em}}
            \textbf{Überladen von Methoden} & \textbf{Überschreiben von Methoden}
            \\
            Überladen von Methoden heißt, dass wir mindestens zwei Methoden vom gleichem Namen
            innerhalb einer Klasse haben. Der Rückgabetyp und die Parameter können verschieden
            sein, aber die Parameterlisten müssen verschieden sein. Eine Methode wird nicht
            ausschließlich durch den Namen identifiziert, sondern auch über die Typen, Anzahl und
            Reihenfolge ihrer Argumente - ihre Signatur. Daher können mehrere Methoden mit
            demselben Namen in einer Klasse vorhanden sein, solange sich die Signaturen
            unterscheiden.
            & Überschreiben von Methoden heißt, dass die Methode aus der direkt abgeleiteten
            Klasse exakt den gleichen Namen, denselben Rückgabetyp (oder vom Subtyp) und
            dieselben Parameter wie die Methode aus der Basisklasse hat.
            \\
            Die gleichen Methodennamen können für semantisch
            \newline
            ähnliche Funktionalitäten auf unterschiedlichen
            \newline
            Datentypen verwendet werden.
            \newline
            (siehe \inlinejava{String.valueOf(...)} für \inlinejava{boolean}, \inlinejava{char}...)                                                                                                                                                                                                                                                                                                                                                                                                                                      & Dadurch erreicht man, dass beim Objekt der abgeleiteten Klasse die neue Implementierung (in der abgeleiteten Klasse) beim Aufruf der Methode aufgerufen wird und nicht die ursprüngliche Methode. Somit kann man bspw. erreichen, dass Methoden, deren Funktionalität in der abgeleiteten Klasse verändert werden soll, einfach überschrieben werden, aber gleichzeitig die anderen Methoden der Oberklasse immer noch verwenden werden können. Die Unterscheidung, welche Methode verwendet wird, wird anhand des dynamischen Typs (= Typ des Objektes) getroffen. Betrachten wir folgende Fälle: Ist der dynamische Typ die abgeleitete Klasse, so wird die Methode von der abgeleiteten Klasse verwendet. Ist er hingegen die Basisklasse, so wird die Methode von der Basisklasse verwendet.
          \end{tabular}
        \end{table}
      \end{enumerate}
    \end{solution}
  \end{task}

  \clearpagesolution

  \begin{task}[credit=\stars{1}{3}]{Brumm, Brumm, Brumm}
    Schreiben Sie eine Klasse \code{Car} zur Repräsentation von Autos, die folgende Anforderungen
    erfüllen soll:

    \begin{itemize}
      \item Ein Auto hat einen Namen vom Typ \inlinejava{String} und einen Kilometerstand
      (\inlinejava{mileage}) vom Typ \inlinejava{double}. Beide Attribute sollen
      \inlinejava{private}, nicht \inlinejava{public}, sein.
      \item Der Konstruktor soll einen \inlinejava{String} als Parameter erhalten, der den Namen
      des Autos angibt. Der Konstruktor soll den Namen des Autos setzen und den Kilometerstand
      auf \inlinejava{0.0} setzen.
      \item Schreiben Sie die Methoden \inlinejava{public double getMileage()} und
      \inlinejava{public String getName()}. Diese liefern die entsprechenden Attribute der Klasse
      \inlinejava{Car} zurück.
      \item Schreiben Sie die Methode \inlinejava{public void drive(double distance)}, die eine
      Distanz in Kilometern als Argument erhält und auf den alten Kilometerstand addiert.
    \end{itemize}

    \begin{solution}
      \lstinputlisting[style=Java, lastline=24]{codes/V2_Solution.java}

      \clearpage

      \lstinputlisting[style=Java, firstline=26, firstnumber=25]{codes/V2_Solution.java}
    \end{solution}
  \end{task}

  \clearpagesolution

  \begin{task}[credit=\stars{1}{3}]{Interfaces}
    Gegeben sei folgendes Interface

    \lstinputlisting[style=Java]{codes/V3_Task.java}

    Schreiben Sie nun eine Klasse \code{C1}, die das Interface \code{I1} implementiert. Sofern im
    Body einer Methode eine Rückgabe erwartet wird, geben sie \inlinejava{null} zurück.

    \begin{solution}
      \lstinputlisting[style=Java]{codes/V3_Solution.java}
    \end{solution}
  \end{task}

  \clearpagesolution

  \begin{task}[credit=\stars{1}{3}]{Vererbung}
    Gegeben seien folgende zwei Klasse, die sich im gleichen Package befinden:

    \lstinputlisting[style=Java]{codes/V4_Task.java}

    Betrachten Sie die \inlinejava{main}-Methode der Klasse \inlinejava{B2}. Auf welche Attribute
    können Sie mit dem Objekt \inlinejava{obj} zugreifen? Welche Methoden können Sie aufrufen und
    welchen Wert geben die Methoden zurück?

    \begin{solution}
      \begin{itemize}
        \item Wir können auf die folgende Attribute mit dem Objekt \inlinejava{obj} zugreifen:
        \begin{itemize}
          \item \inlinejava{f}, \inlinejava{by}, \inlinejava{i} und \inlinejava{d}
        \end{itemize}
        \item Wir können die folgenden Methoden mit dem Objekt \inlinejava{obj} aufrufen:
        \begin{itemize}
          \item \inlinejava{m1}: Sie gibt den Wert \inlinejava{-1} zurück, da die Methode in der
          Klasse \inlinejava{B1} definiert ist und die Klasse \inlinejava{B2} \inlinejava{B1}
          erweitert.
          \item \inlinejava{m2} der Klasse \inlinejava{B2}: Sie gibt den Wert \inlinejava{2}
          zurück, da die Methode \inlinejava{m2} von \inlinejava{B1} in der Klasse
          \inlinejava{B2} überschrieben wurde.
        \end{itemize}
      \end{itemize}
    \end{solution}
  \end{task}

  \begin{task}[credit=\stars{2}{3}]{Gleicher Abstand}
    Schreiben Sie eine Methode \inlinejava{static boolean evenlySpaced(int a, int b, int c)},
    welche genau dann \inlinejava{true} zurückliefert, wenn der Abstand zwischen dem kleinsten
    und dem mittleren Element genauso groß ist wie der Abstand zwischen dem mittleren und dem
    größten Element. Dabei kann jeder der Parameter \inlinejava{a}, \inlinejava{b} oder
    \inlinejava{c} das kleinste, mittlere oder größte Element sein. Die Klasse, zu der die
    Methode gehört, muss nicht implementiert werden

    \begin{solution}
      \lstinputlisting[style=Java]{codes/V5_Solution.java}
    \end{solution}
  \end{task}

  \clearpagesolution

  \begin{task}[credit=\stars{2}{3}]{Zahlen aneinanderreihen}
    Schreiben Sie eine Methode \inlinejava{static int appendIntegers(int[] a)}. Die Methode
    bekommt ein \inlinejava{int}-Array übergeben und liefert eine Zahl zurück, die entsteht wenn
    man alle Zahlen des übergebenen Arrays aneinanderreiht. Der Aufruf \inlinejava{appendIntegers
      (1,2,3)} liefert \inlinejava{123} zurück. Der Aufruf \inlinejava{appendIntegers(43,2,7777)}
    liefert \inlinejava{4327777} zurück. Sie dürfen nur Variablen von Typ \inlinejava{int} in
    ihrer Implementation verwenden, keine \inlinejava{Strings} oder Ähnliches. Schleifen sind
    natürlich erlaubt.

    \begin{solution}
      \lstinputlisting[style=Java]{codes/V6_Solution.java}
    \end{solution}
  \end{task}

  \clearpage

  \begin{task}[credit=\stars{2}{3}]{Zahlen einsortieren}
    Gegeben sei folgende Klasse

    \lstinputlisting[style=Java]{codes/V7_Task.java}

    Erweitern Sie diese Klasse um eine \inlinejava{public}-Klassenmethode \inlinejava{ArrayTuple
    split(double[] a)}. Die Methode liefert ein neues Objekt von Typ \inlinejava{ArrayTuple}
    zurück, in dessen \inlinejava{int}-Array sich alle ganzen Zahlen aus dem übergebenen Array
    \inlinejava{A} befinden. Im \inlinejava{double}-Array befinden sich die restlichen Zahlen
    aus dem übergebenen Array.

    \begin{solution}
      \lstinputlisting[style=Java, lastline=23]{codes/V7_Solution.java}

      \clearpage

      \lstinputlisting[style=Java, firstline=25, firstnumber=24]{codes/V7_Solution.java}
    \end{solution}
  \end{task}

  \begin{task}[credit=\stars{2}{3}]{Statischer und dynamischer Typ}
    \lstinputlisting[style=Java, lastline=26]{codes/V8_Task.java}

    \clearpage

    \lstinputlisting[style=Java, firstline=26, firstnumber=26]{codes/V8_Task.java}

    \begin{note}[title=Hinweis:, color=tuda-orange]
      Nach Zeile \(X\) heißt unmittelbar nach \(X\), noch vor Zeile \(X + 1\).
    \end{note}

    \begin{enumerate}
      [label=(\arabic*)]
      \item Welchen statischen und dynamischen Typ haben \inlinejava{a}, \inlinejava{b} und
      \inlinejava{g} nach Zeile 40?
      \item Welchen statischen und dynamischen Typ hat \inlinejava{a} und welchen Wert hat
      \inlinejava{a.v} nach Zeile 41?
      \item Welchen Wert haben \inlinejava{t} und \inlinejava{b.v} nach Zeile 42?
      \item Welchen statischen und dynamischen Typ haben \inlinejava{a}, \inlinejava{b} und
      welchen Wert hat
      \inlinejava{a v} nach Zeile 44?
      \item Welchen Wert haben \inlinejava{r} und \inlinejava{b.v} nach Zeile 45?
    \end{enumerate}

    \clearpagesolution

    \begin{solution}
      \begin{enumerate}
        [label=(\arabic*)]
        \item Welchen statischen und dynamischen Typ haben \inlinejava{a}, \inlinejava{b} und
        \inlinejava{g} nach Zeile 40?
        \begin{itemize}
          \item Der statische und der dynamische Typ von \inlinejava{a} ist \inlinejava{Alpha}.
          \item Der statische und der dynamische Typ von \inlinejava{b} ist \inlinejava{Beta}.
          \item Der statische und der dynamische Typ von \inlinejava{g} ist \inlinejava{Gamma}.
        \end{itemize}
        \item Welchen statischen und dynamischen Typ hat \inlinejava{a} und welchen Wert hat
        \inlinejava{a.v} nach Zeile 41?
        \begin{itemize}
          \item Der statische Typ von \inlinejava{a} ist \inlinejava{Alpha} und der dynamische
          Typ \inlinejava{Beta}.
          \item \inlinejava{a.v} hat den Wert 2.
        \end{itemize}
        \item Welchen Wert haben \inlinejava{t} und \inlinejava{b.v} nach Zeile 42?
        \begin{itemize}
          \item \inlinejava{t} hat den Wert 11.
          \item \inlinejava{b.v} hat den Wert 4.
        \end{itemize}
        \item Welchen statischen und dynamischen Typ haben \inlinejava{A}, \inlinejava{b} und
        welchen Wert hat
        \inlinejava{a.v} nach Zeile 44?
        \begin{itemize}
          \item Der statische Typ von \inlinejava{a} ist \inlinejava{Alpha} und der dynamische
          Typ \inlinejava{Beta}.
          \item Der statische Typ von \inlinejava{b} ist \inlinejava{Beta} und de dynamische Typ
          \inlinejava{Gamma}.
          \item \inlinejava{a.v} hat den Wert 13.
        \end{itemize}
        \item Welchen Wert haben \inlinejava{r} und \inlinejava{b.v} nach Zeile 45?
        \begin{itemize}
          \item \inlinejava{r} hat den Wert 41.
          \item \inlinejava{b.v} hat den Wert 2.
        \end{itemize}
      \end{enumerate}

      \begin{note}[title=Information:]

        Der statische Typ ist der Typ, der bei der Variablendeklaration angegeben wird. Er ist
        bei der Kompilierung schon bekannt.

        \br

        Der dynamische Typ ist der Typ des tatsächlichen Objekts, der zur Laufzeit bekannt ist
        und kann vom statischen Typ abweichen. Dieser Typ ist entweder gleich dem statischen Typ
        oder ein Subtyp des statischen Typs.

        \br

        Der formale Aufbau sieht folgendermaßen aus:

        \begin{center}
          \inlinejava{<Statischer Typ> Bezeichner = <Dynamischer Typ>}
        \end{center}

        Als Beispiel nehmen wir hierzu \inlinejava{FOPBot}. Als statischen Typ nehmen wir die
        Klasse \inlinejava{Robot}. Somit kann der dynamische Typ entweder \inlinejava{Robot} sein
        oder ein Subtyp dessen.

        \begin{itemize}
          \item \code{Robot robot = new Robot(1, 1, UP, 1);}: In diesem Fall ist der dynamische
          Typ gleich dem
          statischen Typ.
          \item \code{Robot robot = new SymmTurner(1, 1, UP, 1);}: In diesem Fall ist der
          dynamische Typ ein Subtyp
          des statischen Typ.
        \end{itemize}
      \end{note}
    \end{solution}
  \end{task}

  \clearpage

  \begin{task}[credit=\stars{3}{3}]{Klassen, Interfaces und Methoden}
    \begin{subtask}
      Schreiben Sie ein \inlinejava{public}-Interface \inlinejava{A} mit einer Objektmethode
      \inlinejava{m1}, die Rückgabetyp \inlinejava{double}, einen \inlinejava{int}-Parameter
      \inlinejava{n} und einen \textcolor{keywordcolor}{char}-Parameter \code{c} hat.

      \begin{solution}
        \lstinputlisting[style=Java]{codes/V9_1_Solution.java}
      \end{solution}
    \end{subtask}

    \begin{subtask}
      Schreiben Sie ein \inlinejava{public}-Interface \inlinejava{B}, das von \inlinejava{A} erbt
      und zusätzlich eine Objektmethode \inlinejava{m2} hat, die keine Parameter hat und einen
      \inlinejava{String} zurückliefert.

      \begin{solution}
        \lstinputlisting[style=Java]{codes/V9_2_Solution.java}
      \end{solution}
    \end{subtask}

    \clearpagesolution

    \begin{subtask}
      Schreiben Sie eine \inlinejava{public}-Klasse \inlinejava{XY}, die \inlinejava{A}
      implementiert, aber \inlinejava{m1} nicht. Klasse \inlinejava{XY} soll ein
      \inlinejava{protected}-Attribut \inlinejava{p} vom Typ \inlinejava{long} haben sowie einen
      \inlinejava{public}-Konstruktor mit Parameter \inlinejava{q} vom Typ \inlinejava{long}. Der
      Konstruktor soll \inlinejava{p} auf den Wert von \inlinejava{q} setzen. Weiter soll
      \inlinejava{XY} eine \inlinejava{public}-Objektmethode \inlinejava{m3} mit Rückgabetyp
      \inlinejava{void} und Parameter \inlinejava{XY} vom Typ \inlinejava{XY} haben, aber nicht
      implementieren.

      \begin{solution}
        \lstinputlisting[style=Java]{codes/V9_3_Solution.java}
      \end{solution}
    \end{subtask}

    \begin{subtask}
      Schreiben Sie eine \inlinejava{public}-Klasse \inlinejava{YZ}, die von \inlinejava{XY} erbt
      und \inlinejava{B} implementiert. Die Methode \inlinejava{m1} soll \inlinejava{n+c+p}
      zurückliefern und \inlinejava{m2} den String \code{\textcolor{stringcolor}{\grqq
      Hallo\grqq}}. \inlinejava{m3} soll den Wert \inlinejava{p} von \inlinejava{XY}auf den Wert
      \inlinejava{p} des eigenen Objektes addieren. Der Konstruktor von \inlinejava{YZ} ist
      \inlinejava{public}, hat einen \inlinejava{long}-Parameter \inlinejava{r} und ruft damit
      den Konstruktor von \inlinejava{XY} auf.

      \begin{solution}
        \lstinputlisting[style=Java]{codes/V9_4_Solution.java}
      \end{solution}
    \end{subtask}
  \end{task}

  \begin{task}[credit=\stars{3}{3}]{Jedes dritte Element}
    Gegeben sei eine Klasse \inlinejava{X}. Schreiben Sie für diese Klasse die
    \inlinejava{public}-Objektmethode \inlinejava{foo}. Diese hat ein Array \inlinejava{a} von
    Typ \inlinejava{int} als formalen Parameter und liefert ein anderes Array \inlinejava{b}
    vom Typ \inlinejava{int} zurück, das aus \inlinejava{a} entsteht, indem jedes dritte Element
    gelöscht wird, das heißt, die Elemente von \inlinejava{a} an den Indizes \(0, 3, 6, 9, \dots\)
    werden nicht nach \inlinejava{b} kopiert, alle anderen Elemente von \inlinejava{a} werden in
    derselben Reihenfolge, wie sie in \inlinejava{A} stehen, nach \inlinejava{b} kopiert. Weitere
    Elemente hat \inlinejava{b} nicht. Sie dürfen voraussetzen, dass \inlinejava{A} mindestens
    Länge \inlinejava{2} hat und ungleich \inlinejava{null} ist. Sie dürfen einfach Operator
    \inlinejava{=} für das Kopieren von Elementen verwenden.

    \br

    \begin{note}[title=Hinweis:, color=tuda-orange]
      Überlegen Sie sich die Gesetzmäßigkeit, nach der die Indizes \(1, 2, 4, 5, 7, 8, \dots\) in
      \inlinejava{a} auf die Indizes \(0, 1, 2, 3, 4, 5, \dots\) in \inlinejava{b} abzubilden
      sind. Für die Länge von \inlinejava{b} werden Sie eine Fallunterscheidung benötigen, je
      nachdem, welchen Rest \inlinejava{a.length} dividiert durch \inlinejava{3} ergibt. Denken
      Sie auch an die letzten beiden Elemente von \inlinejava{a}.
    \end{note}

    \begin{solution}
      \lstinputlisting[style=Java]{codes/V10_Solution.java}
    \end{solution}
  \end{task}
\end{document}
